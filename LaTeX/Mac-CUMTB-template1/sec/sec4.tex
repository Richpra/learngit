% !Mode::"TeX:UTF-8"
% !TEX encoding = UTF-8
\section{误差分析}
在迭代收敛定理的条件下,可以得到下列误差估计式
\begin{equation}
\lvert x^*-x_k\rvert\leq \frac q {1-q} \lvert x_k-x_{k-1}\rvert
\end{equation}
\begin{equation}
\lvert x^*-x_k\rvert\leq \frac {q^k} {1-q} \lvert x_1-x_0\rvert
\end{equation}
事实上,根据式$(2.2)$,有
\begin{equation}
\lvert x_{k+1}-x_k\rvert\geq\lvert x^*-x_k \rvert-\lvert x^*-x_{k+1}\rvert\geq(1-q)\lvert x^*-x_k\rvert,
\end{equation}
又利用定理2.1的条件$2$得
\begin{equation}
\lvert x_{k+1}-x_k\rvert=\lvert \varphi(x_k)-\varphi (x_{k-1})\rvert\leq\lvert x_k-x_{k+1} \rvert
\end{equation}
于是得到式$(3.1)$,再反复利用上述关系式可导出式$(3.2)$.
式$(3.2)$表明,只要相邻两次迭代值的偏差$\lvert x_k - x_{k-1} \rvert$足够小,就能保证迭代误差$\vert x^*-x_k$满足预先指定的精度,因此对于指定精度$\varepsilon$,经常用条件$\lvert x_k -x_{k-1}\rvert$来控制迭代过程的结束。

