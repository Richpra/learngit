% !Mode::"TeX:UTF-8"
% !TEX encoding = UTF-8
\section{不动点迭代法的算法构造}
\subsection{基本思路}
已知方程$f(x)=0$在区间$[a,b]$内有唯一的实根$x^*$,将上式改写成等价形式$x=\varphi(x)$,取$x_0\in[a,b]$,用递推公式$x_{k+1}=\varphi(x_k)(k=0,1,2,\cdots)$,可以得到序列$x_0,x_1,\cdots$.如果当$k\rightarrow\infty$时,有极限$ \lim_{k\to\infty}x_k=x^*$,则称迭代法收敛于$x^*$,且$x^*=\varphi(x^*)$,则称$\varphi(x)为迭代函数$,$x_{k+1}=\varphi(x_k)$为迭代格式,${x_k}$为迭代序列;当$x_0\neq x^*$时,如果序列${x_k}$在$[a,b]$内无极限,则称迭代法发散。应当指出,为使迭代法有效,必须保证它的收敛性,一个发散的迭代过程,纵使迭代千万次,其结果也是毫无价值的。因而我们必须考虑迭代法的收敛性。
\subsection{收敛定理}
我们应当如何验证迭代格式的收敛性呢?这里就需要引入收敛定理。
\begin{thm}\label{t2.1}
设迭代函数$\varphi(x)$在$[a,b]$上有连续的一阶导数,且:
(1)当$x\in[a,b]$时,$a\leq\varphi(x)\leqb$.
(2)存在正数$q\leq1$(q为利普希茨常数),使对任意$x\in[a,b]$,有$\lvert{\varphi^'}(x)\rightvert\leqq<1$.
则方程$x=\varphi(x)$在区间上存在唯一的根$x^*$,并且对任意初值$x_0\in [a,b]$,由迭代格式$x_{k+1}=\varphi(x_k)$所确定的序列${x_k}$均收敛于${x^*}$.
\end{thm}
\begin{proof}
设$x^*$为方程$x=\varphi(x)$的根,则由微分中值定理有
\begin{equation}
x^*-x_{k+1}=\varphi(x^*)-\varphi(x_k)={\varphi^'}(\varepsilon)(x^*-x_k),
\end{equation}
式中$\varepsilon$是$x^*$与$x_k$中的某一点,于是有
\begin{equation}
\lvert x^*-x_{k+1}\rvert\leq q\lvert\ x^* - x_k \rvert.
\end{equation}
对此反复递推,对迭代误差$\varepsilon_k=\lvert x^* - x_k \rvert$,有
\begin{equation}
\varepsilon_k \leq q^k\varepsilon_0.
\end{equation}
由于$0<q<1$,因而$\varepsilon_k\rightarrow0(k\rightarrow\infty)$,即迭代法收敛.
\end{proof}
\begin{cor}\label{c2.1}
如果$\lvert{\varphi^'}(x)\rvert>1$,则迭代格式$x_{k+1}=\varphi(x_k)$所确定的序列${x_k}$对任意初值$x_0\in [a,b]$发散.
\end{cor}
若存在$x^*$的某个邻域$I={x:\lvert x-x^* \rvert\leq\delta}$,使得迭代过程$x_{k+1}=\varphi(x_k)$对于任意初值$x_0\in I$收敛,则称迭代过程$x_{k+1}=\varphi(x_k)$在根$x^*$的邻近具有局部收敛性.
\begin{thm}\label{t2.2}
设$\varphi(x)$在$x=\varphi(x)的根x^*的邻近有连续的一阶导数,且成立\lvert {\varphi^'}<1 \rvert$,则迭代过程$x_{k+1}=\varphi(x_k)$在根$x^*$附近具有局部收敛性。
\end{thm}
\begin{proof}
由于$\lvert {\varphi^'}(x^*) \rvert < 1$,故存在充分小的邻域$I={x:\lvert x-x^* \rvert\leq\delta}$,使得$\lvert {\varphi^'} \leq q <1\rvert$成立.这里q为某个常数.据微分中值定理,有
\begin{equation}
\varphi(x)-\varphi(x^*)={\varphi^'}(\varepsilon)(x-x^*),
\end{equation}
注意到$\varphi(x^*)=x^*$,又当$x\in I$时,$\varepsilon \in I$,故有
\begin{equation}
\lvert \varphi(x) - x^* \rvert\leq q\lvert x-x^* \rvert\leq\lvert x-x^* \rvert\leq\delta,
\end{equation}
于是由定理1可以断定$x_{k+1}=\varphi(x_k)$对于任意$x_0 \in I$均收敛.
\end{proof}
\subsection{计算步骤}
迭代过程$x_{k+1}=\varphi(x_k)$的计算步骤如下:
(1)选取初始近似值$x_0$;
(2)由方程$f(x)=0$确定迭代函数$\varphi(x)$;
(3)按照迭代格式$x_{k+1}=\varphi(x_k)$计算$x_1=\varphi(x_0)$;
(4)如果$\lvert x_1-x_0\rvert\geq\varepsilon$,则停止计算,否则用$x_1$代替$x_0$重复(3)和(4).
\newpages